\documentclass[12pt,twoside,openany,showtrims,a5paper]{memoir}
\usepackage[utf8]{inputenc}
\usepackage[spanish]{babel}
\usepackage{amsmath}
\usepackage{amsfonts}
\usepackage{amssymb}
\usepackage{tabularx}
\author{Federico García Lorca}
\title{La casa de Bernarda Alba}
\begin{document}
\book{La Casa de Bernarda Alba}
\part{Acto primero}
\begin{table}[hbt!]
\setlength{\tabcolsep}{2pt} %% default is 6pt
\begin{tabular}{l l l}
\multicolumn{2}{c}{Personajes} & Intérpretes \\ 
BERNARDA & 60 años & Margarita Xirgu \\ 
MARÍA JOSEFA (madre de Brda.) & 80 años & Antonia Herrero \\ 
ANGUSTIAS (hija de Bernarda) & 39 años & Teresa Serrador \\ 
MAGDALENA (hija de Bernarda) & 30 años & Carmen Caballero \\ 
AMELIA (hija de Bernarda) & 27 años & Teresa Pradas \\ 
MARTIRIO (hija de Bernarda) & 24 años & Pilar Muñoz \\ 
ADELA (hija de Bernarda) & 20 años & Isabel Pradas \\ 
LA PONCIA (criada) & 60 años & María Gámez \\ 
CRIADA & 50 años & Luz Barrilaro \\ 
PRUDENCIA & 50 años & Emilia Milán \\ 
MUJER 1ª & - - & Susana N. Gómez \\ 
MUJER 2ª & - - & Aída Espí \\ 
MUJER 3ª & - - & María López Silva \\ 
MUJER 4ª & - - & Emilia López \\ 
MUCHACHA & - - & Susana Canales \\ 
\multicolumn{3}{c}{MUJERES DE LUTO} \\ 
\end{tabular} 
\end{table}
\begin{center}
\scshape{Decorados sobre bocetos de SANTIAGO ONTAÑÓN}
\end{center}
\begin{center}
El poeta advierte que estos tres actos tienen la intención de un documental fotográfico
\end{center}

\paragraph{Mujer 1ª}
Los pobres sienten también sus penas.
\paragraph{Bernarda.}
Pero las olvidan delante de un plato de garbanzos.
\end{document}