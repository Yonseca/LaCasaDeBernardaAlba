% !TEX root = LaCasaDeBernardaAlba.tex
\documentclass[a5paper]{stage}
\usepackage[utf8]{inputenc}
\usepackage[spanish]{babel}
\usepackage{amsmath}
\usepackage{amsfonts}
\usepackage{amssymb}
\renewcommand{\theendname}{Fin}

% Macros
\def\Bernarda{\dialog{Bernarda}}
\def\MaríaJosefa{\dialog{MaríaJosefa}}
\def\Angustias{\dialog{Angustias}}
\def\Magdalena{\dialog{Magdalena}}
\def\Amelia{\dialog{Amelia}}
\def\Martirio{\dialog{Martirio}}
\def\Adela{\dialog{Adela}}
\def\Poncia{\dialog{La Poncia}}
\def\Prudencia{\dialog{Prudencia}}
\def\Criada{\dialog{Criada}}
\def\MujerUno{\dialog{Mujer 1ª}}
\def\MujerDos{\dialog{Mujer 2ª}}
\def\MujerTres{\dialog{Mujer 3ª}}
\def\MujerCuatro{\dialog{Mujer 4ª}}
\def\Muchacha{\dialog{Muchacha}}


    \begin{document}
    \author{Federico García Lorca}
    \title{La casa de Bernarda Alba}
    \maketitle
    \begin{castpage}
		\addcharacter{Bernarda}{60 años. }
		\addcharacter{María Josefa}{Madre de Bernarda. 80 años. }
		\addcharacter{Angustias}{Hija de Bernarda. 39 años}
		\addcharacter{Magdalena}{Hija de Bernarda. 30 años}
		\addcharacter{Amelia}{Hija de Bernarda. 27 años}
		\addcharacter{Martirio}{Hija de Bernarda. 24 años}
		\addcharacter{Adela}{Hija de Bernarda. 20 años}
		\addcharacter{La Poncia}{Criada. 60 años}
		\addcharacter{Criada}{50 años}
		\addcharacter{Prudencia}{50 años}
		\addcharacter{Mujer 1ª}{}
		\addcharacter{Mujer 2ª}{}
		\addcharacter{Mujer 3ª}{}
		\addcharacter{Mujer 4ª}{}
		\addcharacter{Muchacha}{}
		\addcharacter{Mujeres de luto}{}
		\begin{center}
	El poeta advierte que estos tres actos tienen la intención de un documental fotográfico
		\end{center}
    \end{castpage}
    \act
    \scene
    \stage{Habitación blanquísima del interior de la casa de Bernarda. Muros gruesos. Puertas en arco con cortinas de yute rematadas en madroños y volantes. Sillas de anea. Cuadros con paisajes inverosímiles de ninfas o reyes de leyenda. Es verano. Un gran silencio umbroso se extiende por la escena. Al levantarse el telón está la escena sola. Se oyen doblar las campanas. Sale la \introduce{Criada}}
    	\Criada{Ya tengo el doble de esas campanas metido entre las sienes.}
    	\Poncia{\charsd{Sale comiendo chorizo y pan}Llevan ya más de dos horas de gori-gori. Han venido curas de todos los pueblos. La iglesia está hermosa. En el primer responso se desmayó la Magdalena.} 
		\Criada{Es la que queda más sola.}
		\Poncia{Era la única que quería la padre. ¡Ay! ¡Gracias a Dios que estamos solas un poquito! Yo he venido a comer}
		\Criada{¡Si te viera Bernarda...!}
		\Poncia{¡Quisiera que ahora, que no come ella, que todas nos muriéramos de hambre! ¡Mandona! ¡Dominanta! ¡Pero se fastidia! Le he abierto la orza de chorizos}
		\Criada{\charsd{Con tristeza, ansiosa}¿Por qué no me das para mi niña, Poncia?}
		\Poncia{Entra y llévate también un puñado de garbanzos.¡Hoy no se dará cuenta!}
		\dialog{Voz}{\charsd{Dentro}¡Bernarda!}
		\Poncia{La vieja. ¿Está bien cerrada?}
		\Criada{Con dos vueltas de llave}
		\Poncia{Pero debes poner también la tranca. Tiene unos dedos como cinco ganzúas}
		\dialog{Voz}{¡Bernarda!}
		\Poncia{\charsd{A voces} ¡Ya viene! \charsd{A la Criada} Limpia bien todo. Si Bernarda no ve relucientes las cosas me arrancará los pocos pelos que me quedan}
		\Criada{¡Qué mujer!}


\end{document}